\section{Impression}
    \setlength{\parindent}{25pt}
    First, we modify the `pte` to read-only permission after calling "pte\_wrprotect(*pte)" and we are sure that  the value has changed. 
    In fact, if we assign a value to not only the `ptr` that is set only-read permission but other variables in c code, 
    the pte's permission resumes after doing that. So we add "ptep\_set\_wrprotect(mm, PtrToUlong((void *)ptr), \&pte)" to our code. 
    The pte doesn't resume anymore. However, we execute "ptr[0]='w'", the result is still the same.\\

    Therefore we introduce it has an mechanism in Linux kernel to avoid pte fault and 
    find an approach named "copy on write" after tracing those functions "handle\_mm\_fault, handle\_pte\_fault, do\_page\_fault". 
    We think we change the page's protection and rewrite the vm\_flags will let cow not to happen. 
    After doing that, we still not get the error message,"segmentation fault" on kernel 3.14.25. 
    Maybe we miss something else so we will continue to find the solution.
\section{Test first}
  \begin{minted}[mathescape,
               numbersep=5pt,
               gobble=2,
               frame=lines,
               framesep=2mm]{s}
               
   typedef unsigned long ULONG;
   typedef unsigned long* ULONG_PTR;
   #define PtrToUlong( p ) ((ULONG)(ULONG_PTR) (p) )
               
  SYSCALL_DEFINE2(nonwritable, char *, ptr, char *, ptr_mod)
  { 
      pgd_t *pgd;
      pud_t *pud;
      pmd_t *pmd;
      pte_t *pte;
  
      struct mm_struct *mm = current->mm;
      printk("from %s\n",ptr);
      printk("from %c\n",*ptr);
      printk("from 0x%lx~0x%lx\n",mm->mmap->vm_start,mm->mmap->vm_end);
      printk("from %lx\n",PtrToUlong((void *)ptr));
  
      pgd = pgd_offset(mm, PtrToUlong((void *)ptr));
      pud = pud_offset(pgd, PtrToUlong((void *)ptr));
      pmd = pmd_offset(pud, PtrToUlong((void *)ptr));
      pte = pte_offset_map(pmd, PtrToUlong((void *)ptr));
  
      if(pte_write(*pte))
          printk(KERN_INFO "pte still writable!!\n");
  
      printk(KERN_INFO "before bit: %lx\n",pte_val(*pte));
  
      set_pte_atomic(pte, __pte(pte_val(pte_wrprotect(*pte))));
  
      printk(KERN_INFO "after bit: %lx\n",pte_val(*pte));
  
      //flush_cache_mm(mm);
      //flush_tlb_mm(mm);
  
      set_pte_atomic(pte, pte_wrprotect(*pte));
      //set_pte_atomic(pte, __pte(pte_val(pte_wrprotect(*pte))));
  
      pgd = pgd_offset(mm, PtrToUlong((void *)ptr));
      pud = pud_offset(pgd, PtrToUlong((void *)ptr));
      pmd = pmd_offset(pud, PtrToUlong((void *)ptr));
      pte = pte_offset_map(pmd, PtrToUlong((void *)ptr));
  
      printk(KERN_INFO "flush bit: %lx\n",pte_val(*pte));
  
  
  
      ptep_set_wrprotect(mm, PtrToUlong((void *)ptr), &pte);
  
  
      return 1;
  }
  \end{minted}
  
  
\section{Test Second}

	After click more href. on Google... Linux kernel have a mechanism call copy-on-write when pte zero or fault.   \\
	
	handle\_mm\_fault -> handle\_pte\_fault \\
	-> \_do\_page\_fault -> (flags \& FAULT\_FLAG\_WRITE)

	  \begin{minted}[mathescape,
	               numbersep=5pt,
	               gobble=2,
	               frame=lines,
	               framesep=2mm]{s}	
  struct mm_struct *mm = current->mm;
  struct vm_area_struct *vma = mm->mmap;

  unsigned long ori_vm_flags, mod_vm_flags, vm_type;
  pgprot_t protect;
  vm_type = 0xF;

  ori_vm_flags = ori_vm_flags & ~(VM_WRITE);
  //mod_vm_flags = mod_vm_flags | ori_vm_flags;
  mod_vm_flags = ori_vm_flags;

  protect = protection_map[vm_type & mod_vm_flags];


  vma->vm_page_prot = protect;
  vma->vm_flags = mod_vm_flags;
	  \end{minted}
