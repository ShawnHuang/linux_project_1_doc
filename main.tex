\documentclass[a4paper,12pt]{report}
%\usepackage[a4paper,paperheight=29.https://www.writelatex.com/1949581nmtgxg#7cm,paperwidth=21cm,left=2.0cm,right=2cm,top=2cm,bottom=2cm,footskip=1.5cm]{geometry}
\usepackage{pstricks}		   % for writeLatex choose xelatex automaticaly
\usepackage{graphicx}		  % image
\usepackage{xeCJK}				% chinese
\usepackage{verbatim} 		 % ascii block
\usepackage{listings} 			% code highlight
\usepackage{minted} 		  % code highlight
\usepackage{graphvizzz}
\setmainfont{Droid Sans}
\setCJKmainfont{AR PL UKai TW}
\setCJKmonofont{AR PL UKai TW}

\title{ Linux Operation System: Project 1}

\begin{document}
\begin{titlepage}
    \centering
    \vfill
    \vfill

    \centering{\Huge Linux Operation System: Project 1}\\
    \vfill
    \vfill
    \begin{verbatim}
         -------------------------------------
        / Group#28                            \
        \ Members: 歐軒慈 凌于翔 黃南雄     /
         -------------------------------------
                   \
                    \
                      .--.           .--.           .--.
                     |o_o |         |o_o |         |o_o |
                     |:_/ |         |:_/ |         |:_/ |
                    //   \ \       //   \ \       //   \ \
                   (|     | )     (|     | )     (|     | )
                  /'\_   _/`\    /'\_   _/`\    /'\_   _/`\
                  \___)=(___/    \___)=(___/    \___)=(___/

     \end{verbatim}
    \vfill
    \vfill
\end{titlepage}

%\begin{tableofcontents}
\tableofcontents
%\end{tableofcontents}


\chapter{Our environment}
	\input{'Our_env.tex'}
    
\chapter{How to compile linux kernel}
 	\input{'How_to_compile_linux_kernel'}

\chapter{Add a new system call in Linux kernel}
 	\input{'Add_syscall.tex'}
    
\chapter{Find the process according to the pid parameter}
 	\input{'Find_with_pid.tex'}
  
\chapter{Dump the process virtual address space areas}
	\input{'Dump_va.tex'}
  
\chapter{Dump the physical frame addresses that the process is using}
	\input{'Dump_pfa.tex'}
  
\chapter{Optional Bonus Point}
  \input{'bonus.tex'}
  %\includegraphics[scale=0.5]{figure01.jpg}
\end{document}
